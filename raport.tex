\documentclass{article}
\usepackage[T1]{fontenc}
\usepackage[utf8]{inputenc}
\usepackage{amsmath}

\title{Raport z negocjacji}
\author{Marek Kacprzak}
\date{11 października 2024}

\begin{document}

\maketitle

\tableofcontents

\section{Podsumowanie}
Zadaniem zespołu jest opracowanie aplikacji, która pozwala na wgląd w osiągnięcia naukowe pracowników akademickich. Aplikacja będzie prezentować różnorodne dane, takie jak obszary badawcze, publikacje oraz wskaźniki bibliometryczne poszczególnych naukowców. Dodatkowo, system umożliwi porównywanie dorobku między naukowcami.

\section{Technologia}
Aplikacja będzie działać w przeglądarce internetowej i zostanie oparta na następujących technologiach:
\begin{itemize}
  \item \textbf{Golang} - Język programowania do tworzenia API.
  \item \textbf{React} - Biblioteka JavaScript do tworzenia interfejsów użytkownika.
  \item \textbf{Python} - Język programowania do tworzenia skryptów scrapujących.
  \item \textbf{PostgreSQL} - System zarządzania relacyjnymi bazami danych.
  \item \textbf{DigitalOcean} - Dostawca usług chmurowych.
  \item \textbf{Docker} - Narzędzie do virtualizacji aplikacji.
\end{itemize}

\section{Opis działania aplikacji}
\subsection{Główne widoki}
\begin{itemize}
  \item \textbf{Strona główna} - Zawiera wyszukiwarkę pozwalającą na znalezienie pracowników naukowych według różnych kryteriów. Na stronie tej wyświetlani będą też naukowcy o najwyższej liczbie publikacji oraz najlepiej wypadający pod względem wskaźników bibliometrycznych.
  \item \textbf{Widok pracownika} - Prezentuje szczegółowe informacje o naukowcu, w tym jego obszary badań, wskaźniki bibliometryczne oraz publikacje. Użytkownik może również przejść do porównania z innymi naukowcami.
  \item \textbf{Widok porównania} - Umożliwia zestawienie dorobku naukowców według różnych filtrów, takich jak stopień naukowy czy dziedzina badań. Aplikacja wygeneruje graficzną wizualizację wyników porównania.
\end{itemize}

\subsection{Logika działania}
\begin{itemize}
  \item Aplikacja będzie regularnie pobierać dane ze strony \texttt{bw.sggw.edu.pl} i aktualizować swoją bazę danych.
  \item Po wpisaniu odpowiednich informacji, użytkownik będzie mógł wyszukać naukowca, a dane zostaną pobrane z bazy i wyświetlone w widoku naukowca.
  \item Aplikacja pozwoli także na porównanie dorobku kilku naukowców, z odpowiednią wizualizacją wyników.
\end{itemize}

\subsection{Mechanizmy}
\begin{itemize}
  \item \textbf{Mechanizm pobierania danych} - Odpowiada za regularne zbieranie danych ze strony \texttt{bw.sggw.edu.pl}.
  \item \textbf{Mechanizm aktualizacji danych} - Umożliwia aktualizację bazy danych w chmurze po każdej operacji pobierania danych.
  \item \textbf{Mechanizm wyszukiwania} - Pozwala na wyszukiwanie pracowników według filtrów z tolerancją na błędy użytkownika.
  \item \textbf{Mechanizm wizualizacji} - Przetwarza dane z bazy i przedstawia je w formie graficznych podsumowań i porównań.
\end{itemize}

\subsection{Możliwe wyzwania i rozwiązania}
\begin{itemize}
  \item \textbf{Problemy z pobieraniem danych} - Zmiana struktury strony może spowodować, że skrypty przestaną działać. Możliwe rozwiązanie to wprowadzenie systemu powiadomień dla osób odpowiedzialnych za utrzymanie aplikacji.
  \item \textbf{Trudności w porównywaniu dorobku} - Porównanie naukowców z różnych dziedzin może być trudne ze względu na różnice w ich wskaźnikach bibliometrycznych. Rozwiązaniem może być dodanie sekcji wyjaśniającej metodę obliczania wskaźników i specyfikę różnych dziedzin naukowych.
  \item \textbf{Wydajność przy dużych zbiorach danych} - Aplikacja może działać wolno przy dużej liczbie danych. Konieczne jest wdrożenie optymalnych algorytmów wyszukiwania i sortowania.
\end{itemize}

\subsection{Pytania}
\begin{enumerate}
  \item Jak najlepiej porównywać naukowców z różnych dziedzin?
  \item Kim będą główni użytkownicy aplikacji?
  \item Czy aplikacja będzie dostępna dla ogółu, czy tylko dla klienta?
  \item Czy interfejs powinien obsługiwać różne wersje językowe?
  \item Czy aplikacja powinna zbierać dane o wyszukiwaniach i generować statystyki?
  \item Czy interfejs powinien posiadać określoną paletę kolorów?
  \item Jakie wskaźniki bibliometryczne są najistotniejsze?
  \item Czy znaczenie wskaźników zależy od dziedziny naukowca?
  \item Czy aplikacja powinna umożliwiać porównanie więcej niż dwóch naukowców jednocześnie?
  \item Jak często baza danych powinna być aktualizowana?
\end{enumerate}

\section{Sekcja techniczna}
\subsection{Podział projektu}
Na podstawie wymagań projektowych podzielono projekt na poniższe komponenty:

\subsubsection{Scraper}
\begin{itemize}
  \item \textbf{Python:} \textbf{Beautiful Soup} lub \textbf{Scrapy}
  \item \textbf{JavaScript:} \textbf{Puppeteer}
  \item Scraper będzie odpowiadał za przeglądanie strony i zbieranie danych, które następnie zostaną zapisane w bazie danych.
\end{itemize}

\subsubsection{Baza danych}
\begin{itemize}
  \item \textbf{PostgreSQL} - Brak dodatkowych uwag.
\end{itemize}

\subsubsection{API}
\begin{itemize}
  \item API będzie pośredniczyć między bazą danych a frontendem użytkownika.
  \item Możemy napisać API w Pythonie, Go, Rust, lub, ewentualnie, w C\#, choć ostatnia opcja nie jest preferowana.
\end{itemize}

\subsubsection{Frontend}
\begin{itemize}
  \item Możliwe są dwa podejścia: tradycyjne z użyciem HTML, CSS i JS lub nowoczesne z wykorzystaniem frameworka, np. \textbf{React}, \textbf{Angular} czy \textbf{Vue}.
  \item Frontend będzie odpowiedzialny za prezentację danych z bazy i umożliwi użytkownikowi filtrowanie informacji.
\end{itemize}

\subsubsection{Dokumentacja}
\begin{itemize}
  \item Dokumentację można generować automatycznie tam, gdzie jest to obsługiwane, np. za pomocą \textbf{OpenAPI}, lub przygotować ręcznie w plikach \texttt{.yml}.
\end{itemize}

\subsubsection{Testowanie}
\begin{itemize}
  \item \textbf{Testowanie API} - Testy jednostkowe.
  \item \textbf{Testowanie frontendu} - Testy integracyjne.
  \item \textbf{Testowanie scrapera} - Sposób testowania będzie ustalony później.
\end{itemize}

\end{document}
